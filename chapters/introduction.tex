\chapter{\label{cha:introduction}Introduction}


\section{Basics}

\subsection{\label{sec:chapters}Chapters and sections}

Place each of your chapters in a separate file in the \verb!chapters/!
directory and make sure you add them to both the \verb!Makefile! and
\verb!thesis.tex!.

It is a good idea to add a sensible label to every chapter and
significant sections, so that you can easily cross-reference them
later. For example, this is Section~\ref{sec:chapters} in chapter
\ref{cha:introduction} and the next chapter will start on page
\pageref{cha:example}.

Also note that each appendix of your thesis should be typeset like an
ordinary chapter.

\subsection{Comments}

Comments can be added to the thesis using the 
\todo{This is a margin comment} \verb!todo! command, as shown in the 
right margin. You can also add inline comments:

\todo[inline, color=green!40]{Remember to remove comments before you submit!.}

\subsection{Citations}

Use \verb!cite! to enter a citation into the main text of your thesis,
e.g.:~\cite{Roscoe+97}. You can list a number of citations at once, by
separating the references inside the \verb!cite! directive with
commas, like this:~\cite{Geurts90,Martin96,Welch+99}. Do \emph{not}
place spaces inside a \verb!cite!ation. You should place your BibTeX
references in files inside the \verb!refs/! directory and add the file
names to the file \verb!bib.tex!. A list of references will be
automatically generated and placed after the main chapters of your
thesis, just before any appendices.


\subsection{Lists} 

\LaTeX{} has three types of list: itemized lists, enumerated lists and
description lists. Because most of this thesis template is
double-spaced, in order to make lists readable, you should use the
\verb!compresslist! directive after each \verb!being{listtype}!
statement. 

\subsubsection{Itemized lists}

\begin{itemize}\compresslist{}
\item foo
\item bar
\end{itemize}

\subsubsection{Enumerated lists}

\begin{enumerate}\compresslist{}
\item foo
\item bar
\end{enumerate}

\subsubsection{Description lists}

\begin{description}\compresslist{}
\item [foo] bar
\item [baz] arg
\end{description}

\subsection{Quotations}

This template provides three ways to typeset quotations and similar
content: footnotes\wlvfootnote{Example footnote.}, ordinary quotes,
with a citation, and ``bare'' quotes, without a citation. Ordinary
quotations are typeset with the \verb!wlvquote! directive and look
like this:

\wlvquote{Lorem ipsum dolor sit amet, consectetur adipisicing elit,
  sed do eiusmod tempor incididunt ut labore et dolore magna
  aliqua. Ut enim ad minim veniam, quis nostrud exercitation ullamco
  laboris nisi ut aliquip ex ea commodo consequat. Duis aute irure
  dolor in reprehenderit in voluptate velit esse cillum dolore eu
  fugiat nulla pariatur. Excepteur sint occaecat cupidatat non
  proident, sunt in culpa qui officia deserunt mollit anim id est
  laborum.}{SugiharGupta08}

``Bare'' quotations, without a citation are typeset with
\verb!wlvbarequote!, and look essentially the same:

\wlvbarequote{Lorem ipsum dolor sit amet, consectetur adipisicing
  elit, sed do eiusmod tempor incididunt ut labore et dolore magna
  aliqua. Ut enim ad minim veniam, quis nostrud exercitation ullamco
  laboris nisi ut aliquip ex ea commodo consequat. Duis aute irure
  dolor in reprehenderit in voluptate velit esse cillum dolore eu
  fugiat nulla pariatur. Excepteur sint occaecat cupidatat non
  proident, sunt in culpa qui officia deserunt mollit anim id est
  laborum.}


\subsection{Figures}

\wlvfig[1.0]{mandelbrot}{This is a caption for a PNG figure.}

You can import any type of graphics file (jpeg, png, etc.) into your
document using the \verb!wlvfig! command. You should place your
figures in the \verb!figures/! directory and add them to
\verb!Makefile!. Figures will automatically be added to the list of
figures at the beginning of the thesis using the caption, which is the
second argument to \verb!wlvfig!.


Figures, such as figure~\ref{fig:mandelbrot} on 
page~\pageref{fig:mandelbrot} are automatically given labels of the form
\verb!fig:FILENAME!. This is the argument that you need to pass to
\verb!ref! or \verb!pageref!. 


Figure~\ref{fig:mandelbrot} was inserted with the command:
\begin{verbatim}
\wlvfig[1.0]{mandelbrot}{This is a caption for a PNG figure.}
\end{verbatim}

The corresponding graphics file is \verb!figures/mandelbrot.png! and
we can refer to the figure with the command
\verb!\ref{fig:mandelbrot}!. The first argument, \verb!1.0! tells
\LaTeX{} to scale the figure to $0.1$ times the original size (keeping
the aspect ratio fixed).

\subsubsection{Placement}

Figures are placed either where you put them in the document, if this
is possible, or failing that, at the top or bottom of the current
page.

 
\subsection{Tables}

Tables, such as~\ref{tab:times} can be typeset with the \verb!wlvtab!
directive. The second argument is a label which should be passed to
\verb!ref! as \verb!tab:LABEL! for cross-referencing, the third
argument is the table caption and the fourth is the contents of the
table itself. Tables will be automatically added to the list of
tables at the start of the thesis.

The first argument to \verb!wlvtab! tells \LaTeX{} how to lay the
table out on the page. Table~\ref{tab:times} was given the directive
\verb!|l||c|c|! which means ``One column left-justified, two vertical
bars, one column centered, one vertical bar, one column centered''. If
you use emacs, you will find \verb!M-x align-regexp! to be very
helpful in making \LaTeX{} tables readable. 

\wlvtab{r|rrrrr} {times} {Table example}  {
  \toprule{}
    & 1 & 2 & 3 & 4 & 5 \\
  \midrule{}
  1 & 2.36 & 1.08 & -0.49 & -0.82 & -0.65 \\
  2 & -0.68 & -1.13 & -0.42 & -0.72 & 1.51 \\
  3 & -1.00 & 0.02 & -0.54 & 0.31 & 1.28 \\
  4 & -0.99 & -0.54 & 0.97 & -1.12 & 0.59 \\
  5 & -2.35 & -0.29 & -0.53 & 0.30 & -0.30 \\
  6 & -0.10 & 0.06 & -0.85 & 0.10 & -0.60 \\
  7 & 1.28 & -0.46 & 1.33 & -0.66 & -1.80 \\
  8 & 0.80 & 0.46 & 1.37 & 1.73 & 1.93 \\
  9 & -0.75 & 0.28 & 0.51 & 0.19 & 0.58 \\
  10 & -1.64 & -0.12 & -1.17 & -0.10 & -0.04 \\
   \bottomrule{}
}


\section{Typesetting mathematics}

\LaTeX{} has built in support for typesetting mathematics, but in a
longer, more structured document you will want to use specific
environments. This template provides support for the following:
\verb!theorem!, \verb!lemma!, \verb!definition!, \verb!case!, which
can be used as below:

\begin{theorem}
\label{thm:evens}
Square roots of even numbers are not rational.
\end{theorem}

\begin{lemma}
\label{thm:two}
The square root of two is not rational.
\end{lemma}

\subsection{Equations}

Numbered equations can be typeset with the \verb!wlveqn! directive,
such as equation~\ref{eqn:secondlaw} on 
page~\pageref{eqn:secondlaw}. The first argument to this should be a label,
which can be used for cross-referencing in the format
\verb!eqn:LABEL!. The second argument to \verb!wlveqn! should be the
equation itself.

\wlveqn{secondlaw}{F=ma}


\section{Thesis statement}


In the introduction of your thesis you will most likely want to define
a \emph{thesis statement}, or hypothesis, or similar. You will want to
refer back to this several times, and particularly in your
conclusions. This template provides specific support for writing a
thesis statement, using the \verb!wlvthesis! directive, the first
argument to which should be a label, the second should be the
statement itself:

\wlvthesis{thesisstmt}{The thesis of this work is that \LaTeX{} rocks!
  We justify this statement by\ldots{}}


\section{Typesetting software}

There are three ways to typeset code using the \verb!listings!
  package. You can typeset code \emph{inline}, using the
  \verb!\lstinline! directive which will result in the following:
  \lstinline[language=Python]$print('hello world!')$ inline
  expression.

Alternatively, you can use \verb!lstlisting! in a begin/end block with
a caption. In which case, you can add a caption which will also appear
in the list of listings at the beginning of the thesis. You can
exclude the listing from the list of listings by adding \verb!nolol!
to the list of directives in the square brackets (look at the file 
\verb!chapters/introduction.tex! for details):

\begin{lstlisting}[language=Python,label={lst:pythonpi},caption={[Redefining a constant of the Universe in Python]Redefining a constant of the Universe in Python}]
>>> import math
>>> math.pi = 4.0
>>> 2.0 * math.pi ** 2
32.0
>>> 
\end{lstlisting}


Lastly, and most useful of all, you can use \verb!lstinputlisting! to
include code from a separate file on disk. This means that you can
edit your code without having to manually merge your efforts back into
the thesis chapters:

\lstinputlisting[language=python,caption={[This caption will appear in the list of listings]This caption will appear under the listing}]{src/hello.py}

To add extra programming languages to listings, edit the
\verb!lstloadlanguages! directive in the file \verb!thesis.tex!. If
you want more fine-grained control over the way that listings are
typeset, including control over syntax highlighting, edit the file
\verb!tango.tex!

Lastly, it is a good idea to read the listings manual, which you can
find online here:
\url{http://mirror.ctan.org/macros/latex/contrib/listings/listings.pdf}


%%% The following are used by emacs, and similar:

%%% Local Variables: ***
%%% TeX-master: "../thesis.tex"  ***
%%% End: ***
